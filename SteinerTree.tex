\documentclass{article}
\usepackage[utf8]{inputenc}

\usepackage{amsmath,amssymb,amsthm}
\usepackage{comment}
\usepackage{hyperref}
\usepackage{pgf,tikz}
\usepackage{showkeys}

\usetikzlibrary{arrows}
\usetikzlibrary{arrows.meta}

\newcommand{\RR}{\mathbb R}
\newcommand{\NN}{\mathbb N}
\newcommand{\ZZ}{\mathbb Z}

\renewcommand{\H}{\mathcal H}
\newcommand{\eps}{\varepsilon}

\newcommand{\loc}{\mathrm{loc}}

\renewcommand{\vec}[1]{\mathbf{#1}}
\newcommand{\abs}[1]{\left\vert #1 \right\vert}
\newcommand{\Abs}[1]{\left\Vert #1 \right\Vert}
\newcommand{\enclose}[1]{\left(#1\right)}
\newcommand{\Enclose}[1]{\left[#1\right]}
\newcommand{\ENCLOSE}[1]{\left\{#1\right\}}
\newcommand{\weakstarto}{{\stackrel{*}{\rightharpoonup}}}

\newcommand{\St}{\mathrm{St}}
\newcommand{\M}{\mathcal{M}}
\renewcommand{\H}{\mathcal{H}}
\newcommand{\dist}{\mathrm{dist}}
\newcommand{\co}{\mathrm{co}}
\renewcommand{\S}{\mathcal{S}}

\newtheorem{theorem}{Theorem}[section]
\newtheorem{proposition}[theorem]{Proposition}
\newtheorem{conjecture}[theorem]{Conjecture}
\newtheorem{lemma}[theorem]{Lemma}
\newtheorem{corollary}[theorem]{Corollary}
\newtheorem{assumption}[theorem]{Assumption}
\newtheorem{problem}[theorem]{Problem}
\newtheorem{openpb}[theorem]{Open Problem}
\theoremstyle{definition}
\newtheorem{definition}[theorem]{Definition}
\theoremstyle{remark}
\newtheorem{remark}[theorem]{Remark}
\newtheorem{example}[theorem]{Example}


\title{Solution to the Steiner problem on an infinite self-similar set}
\author{Paolini Stepanov}

\begin{document}

\maketitle
Let $\St(C)$ be the family of all compact sets $S$ such that 
$S\cup C$ is connected.
Let $\M(C)$ be the subset of $\St(C)$ of sets $S$ 
with minimal length $\H^1(S)$.

Let $A$ be the Cantor-like set you know which one: 
$f_j(z) = 1 + \theta z$,
$\theta = \lambda e^{i \frac \pi 3}$.
$A=A_1\cup A_2$ 
with $A_j = f_j(A)$.

Let $Y=\ENCLOSE{(x,y)\colon x=0}$ be the $y$-axis,
let $Y_j = f_j(Y)$.
Consider the point $P=(1+\frac\lambda 2,0)$ 
and the lines $\tilde Y_j$ parallel to $Y_j$ 
passing through $P$.

Let $\Sigma\in \St(Y\cup A)$ be the binary self-similar 
tree you know which one.
We claim that $\M(Y\cup A)=\ENCLOSE{\Sigma}$.

\textbf{Ricordare. Dire che $S$ è connesso. 
Dire che non importa se $Y$ non è limitato.
Dire che dati due punti di $S$ esiste un unico arco che li connette.
Dire cos'è un arco. }

\begin{lemma}\label{lm:exists}
  If $C$ be a compact subset of $\RR^2$ and $\ell$ be a line.
  Then $\M(\ell\cup C)\neq \emptyset$.
  
  If $C$ is also totally disconnected, $C\cap \ell=\emptyset$ 
  and $S\in \M(\ell\cup C)$  
  then $\bar S \supset C$.

  If moreover $\H^1(S)<+\infty$ and $\H^1(C)=0$ 
  then $\bar S\in \M(\ell\cup C)$.
\end{lemma}
\begin{proof}
  Take a compact connected segment $\sigma\subset \ell$ such that 
  the orthogonal projection of $C$ on $\ell$ is contained 
  in $\sigma$. 
  Then $\sigma\cup C$ is compact and by 
  \cite{PaoSte} we know that 
  $\M(\sigma \cup C)\neq \emptyset$.
  For all $\Sigma \in \St(\ell\cup C)$ we have that its projection 
  $\tilde \Sigma$ 
  onto the convex hull of $\sigma\cup C$ is in $\St(\sigma\cup C)$
  and $\H^1(\tilde \Sigma)\le \H^1(\Sigma)$.
  Hence $\M(\ell\cup C)\supset \M(\sigma\cup C)\neq \emptyset$ as claimed.

  To prove the second part note that if $C$ is totally disconnected 
  then for every $x\in C$ one has $x\in \bar S$ otherwise 
  $S\cup C\cup \ell$ cannot be connected. 
  Therefore $C\subset \bar S$. 
  
  For the last claim notice that $\bar S\cap \ell$ is a finite set.
  In fact the number of connected components of $S$ arriving at $\ell$ 
  is finite since every such component must touch also points of $C$ 
  and hence has length at least $\dist(C,\ell)$.  
  Hence they cannot be infinitely many otherwise 
  we would have $\H^1(S)=+\infty$.
  By \cite{PaoSte}) the closure of every connected component 
  of $S\in\M(C\cup \sigma)$ 
  has at most one point on $\sigma$, showing that the set 
  $W:=\bar S\cap \sigma = \bar S \cap \ell$ 
  is finite.

  Moreover $\bar S \setminus S \subset C\cup W$ because 
  $S\cup C\cup \sigma$ is compact by~\cite{PaoSte}.
  When $H^1(C)=0$ this implies $\H^1(\bar S)=\H^(S)$ and hence 
  $\bar S\in \M(\ell\cup C)$.
\end{proof}

\begin{lemma}\label{lm:angle}
Let $S\in \M(\ENCLOSE{H} \cup C)$ with $C$ a compact set, $H\not \in C$.
Then there exists a point $T$ such that $[HT]\subset \bar S$ 
and either $T\in C$ or $S$ is a triple point in a neighbourhood of $T$.

Moreover if $C\subset B_\rho(P)$ for some point $P$ and radius $\rho > 0$
then $\bar S\setminus B_r(P) = [HT]\setminus B_r(P)$ 
for $r = 2\rho/\sqrt 3$.
\end{lemma}

\begin{lemma}\label{lm:base}
  Let $\ell$ be a line, $\rho>0$, 
  $r=2\rho/\sqrt 3$.
  Let $P$ be a point such that $\dist(P,\ell)>2\rho$
  and let $C$ be a compact subset of $\overline{B_\rho(P)}$.
  Let $S \in \M(\ell\cup C)$ 
  (such $S$ exists in view of Lemma~\ref{lm:exists}).
  Then there exists a point $Q\in \partial B_r(P)$ 
  and a point $H\in \ell$
  such that $\bar S\setminus B_r(P) = [H,Q]$.
  Moreover $[H,Q]$ is perpendicular to $\ell$.
\end{lemma}
\begin{proof}
  \emph{Step 1.} We claim that for any $H\in \ell$ for any $S\in \M(\ENCLOSE{H} \cup C)$
  one has that $\bar S\setminus B_r(P)$ is a straight line segment $[H,Q]$ 
  for some $Q\in \partial B_r(P)$.

  In fact let $S_1$ be the only connected component of $S$ such that 
  $H\in \bar S_1$ (see \cite{PaoSte}).
  By \cite{PaoSte} we know that $\bar S_1\setminus B_\rho(P)$ is locally 
  a finite Steiner tree composed by straight line segments possibly joined 
  by triple points.
  If the claim were false, there would be a triple point $T$
  in $\bar S_1\setminus B_r(P)$ with branchings at 120 degrees dividing 
  the plane in three angles.
  Since $\abs{TP}>r$ we have that $B_r(P)$ is in one of those angles 
  which we call $\alpha$.



   


  We know (see \cite{PaoSte}) that $S$ is locally a finite Steiner tree 
  outside of any small neighbourhood of $\ell\cup C$.
  
  We claim that $S\setminus B_\rho(P)$ belongs to 
  a single connected component of $S$. 
  In fact, if not, there are two different connected components 
  $S_1$ and $S_2$ of $S$ which have some points outside $B_\rho(P)$. 
  We know (see \cite{PaoSte}) that $\bar S_1$ and $\bar S_2$ have 
  at most a single end-point on $\ell$. 
  Let $\tilde S_1$ be the connected component of
  
  In fact suppose by contradiction that there are two different 
  points $H_1$ and $H_2$ in $\bar S\cap \ell$. 
  If $H_1$ and $H_2$ are in the closure of the same connected component 
  of $S$, then removing from $S$ a small neighbourhood of $H_2$ 
  consider the connected components $S_1$ and $S_2$ 
  of $S$ such that $H_1\in \bar S_1$ and $H_2\in \bar S_2$.

  Hence, if $S\setminus B_r(P)$ is not a line segment,
  there is a triple point $T$ 
  Otherwise $S$ has a triple point outside $B_r(P)$.
  So the 120 degree angle encloses the whole ball $B_\rho(P)$ 
  and a projection would decrease the length.
\end{proof}
  
\begin{lemma}\label{lm:01}
  Let $P$ and $A$ be defined as above,
  $R=\frac{2\lambda}{\sqrt 3}$.
  Then $A\subset B_\lambda(P)$ and 
  given any $S\in \M(\ENCLOSE{T}\cup A)$ 
  for some $T\not \in B_R(P)$
  we have that $S\setminus B_R(P)$ is a straight line segment.
\end{lemma}
\begin{proof}
  Follows from Lemma \ref{lm:base} with $C:=A$.
\end{proof}

\begin{lemma} 
  For every $\Sigma \in \M(Y\cup A)$ one has that $\bar\Sigma\in \M(Y\cup A)$.
  Moreover, $\bar\Sigma$ is connected and contains $A$.
\end{lemma}

\begin{lemma}\label{lm:tripod}
  Let $\nu_1,\nu_2,\nu_3$ be three unit vectors $\RR^2$
  with $\nu_1+\nu_2+\nu_3=0$. 
  Let $Y_j$ be a line perpendicular to $\nu_j$ for $j=1,2,3$.
  Let $T$ be any point in $\RR^2$ and let $H_j$ be the orthogonal 
  projection of $T$ on $Y_j$.
  Define $d_j(T) = (T-H_j)\cdot \nu_j$
  be the signed distance of $T$ from $Y_j$.
  Then $d_1(T) + d_2(T) + d_3(T)$ is constant (independent of $T$).

  In paarticular if $Y_1,Y_2,Y_3$ are three lines in $\RR^2$
  forming angles of $60$ degrees so that 
  the three pairwise intersections identify
  an equilateral triangle with sides of length $\ell$.
  Let $T$ be any point and let $d_i(T)$ be 
  the signed distance of $T$ from $Y_i$
  with positive sign when $T$ is inside the triangle.
  Then $d_1(T) + d_2(T) + d_3(T) = \frac{\sqrt 3}{2}\ell$.
\end{lemma}
\begin{proof}
  If we move one of the lines parallel to itself by an amount $\delta$ 
  then the sum $d_1(T)+d_2(T)+d_3(T)$ changes by $\delta$, independent 
  of $T$.
  Therefore, without loss of generality, we may assume that the three lines 
  intersect in the origin.
  In this case $d_j(T) = T\cdot \nu_j$ and hence 
  $d_1(T)+d_2(T)+d_3(T) = T\cdot (\nu_1+\nu_2+\nu_3) = 0$
  concluding the proof of the first claim.

  For the second claim just notice that $d_1(T)+d_2(T)+d_3(T)$
  is constant (by the first claim) and hence is equal to the value 
  obtained when $T$ is one of vertices of the triangle, in which case 
  two distances are $0$ and the third one is equal to the height of the 
  triangle which is the claimed value.
\end{proof}

\begin{lemma}\label{lm:precedente1}
One has
\begin{align*}
    \dist(Y,A_1) & \ge 
    1+ \frac{\lambda} 2 
    - \frac{\lambda^2(1+2\lambda)}{2(1-\lambda^2)}\\
    \dist(A_1,A_2) & 
    \ge 
    \sqrt 3 \lambda - \sqrt 3 \frac{\lambda^3}{1-\lambda}.
\end{align*}
\end{lemma}
\begin{proof}
  \begin{align*}
  \dist(Y,A_1) &\ge 1+ \frac \lambda 2 
     - \enclose{\frac{\lambda^2}{2} 
      + \lambda^3 
      + \frac{\lambda^4 }{2}
      + \lambda^5
      + \frac{\lambda^6}{4} + \dots}\\
      &= 1 + \frac \lambda 2 
      - \frac{1}{2}\frac{\lambda^2}{1-\lambda^2}
      - \frac{\lambda^3}{1-\lambda^2}\\
      &= 1+ \frac{\lambda} 2 
      - \frac{\lambda^2(1+2\lambda)}{2(1-\lambda^2)}
  \end{align*}
  while 
  \begin{align*}
    \dist(A_1,A_2)
    &\ge 2\enclose{\frac{\sqrt 3}{2} \lambda 
      -\frac{\sqrt 3}{2}\enclose{\lambda^3 + \lambda^4 + \dots}
     }\\
    &= \sqrt 3 \lambda -\sqrt 3 \frac{\lambda^3}{1-\lambda}.
  \end{align*}
\end{proof}

\begin{lemma}\label{lm:envelope}
Let $\Sigma\in \M(\ENCLOSE{T}\cup X)$
for some compact set $X$ contained in a horizontal strip 
$E=\ENCLOSE{(x,y)\colon \abs{y}\le d}$ of width $2d>0$.
Suppose also that there is some $T'\in \Sigma$, $T'\neq T$ 
such that $[T,T']$ is horizontal.
Then $T\subset E$.
\end{lemma}
%
\begin{proof}
  Suppose by contradiction that $T=(x_T,y_T)$ is outside $E$.
  For example $y_T>d$. 
  Then the convex hull of $\ENCLOSE{T}\cup X$ has a single 
  point, which is $T$, on the line $\ENCLOSE{y=y_T}$. 
  This is in contradiction with the fact that $T'\in \Sigma$ 
  has the same $y$-coordinate as $T$.
\end{proof}
\begin{figure}
  \begin{center}
  \begin{tikzpicture}[line cap=round,line join=round,>=triangle 45,x=4.0cm,y=4.0cm]
    \clip(-0.2,-0.7) rectangle (1.84,0.92);
    \draw (1.0,0.57) node[anchor=north west] {\includegraphics[width=1cm]{blob1.png}};
    \draw (1.19,0.4) node[anchor=north west] {$A_1$};
    \draw (1.05,-0.12) node[anchor=north west] {\includegraphics[width=1cm]{blob2.png}};
    \draw (1.22,-0.28) node[anchor=north west] {$A_2$};
    \draw (1.2,0.35)-- (1,0);
    \draw (1,0)-- (1.2,-0.35);
    \draw [line width=1.2pt] (0,0)-- (1,0);
    \draw (0,0.58)-- (0,-0.58);
    \draw [domain=-0.2:1.84] plot(\x,{(-0.58--0.58*\x)/1});
    \draw [domain=-0.2:1.84] plot(\x,{(-0.58--0.58*\x)/-1});
    \draw (0,0)-- (1,0);
    \draw [line width=1.2pt] (1.2,0.35)-- (1,0);
    \draw [line width=1.2pt](1,0)-- (1.2,-0.35);
    \draw (0.1,-0.7) -- (0.1,0.92);
    \draw [line width=1.2pt,color=blue] (1.15,0.37)-- (1,0.1);
    \draw [line width=1.2pt,color=blue] (1,0.1)-- (1.21,-0.27);
    \draw [line width=1.2pt,color=blue] (1,0.1)-- (0.1,0.1);
    \draw [dash pattern=on 1pt off 1pt,domain=-0.2:1.84] plot(\x,{(-0.68--0.58*\x)/-1});
    \draw [dash pattern=on 1pt off 1pt,domain=-0.2:1.84] plot(\x,{(-0.47--0.58*\x)/1});
    \draw (0.25,0.1)-- (0.25,0);
    \draw (0.1,0.22)-- (0,0.22);
    \draw [{Latex[length=1mm]}-{Latex[length=1mm]}] (0,0.22) -- (0.1,0.22);
    \draw [{Latex[length=1mm]}-{Latex[length=1mm]}] (0.25,0.1) -- (0.25,0);
    \draw [{Latex[length=1mm]}-{Latex[length=1mm]}] (0.67,0.29) -- (0.63,0.22);
    \draw [{Latex[length=1mm]}-{Latex[length=1mm]}] (0.65,-0.1) -- (0.7,-0.18);
    \draw [{Latex[length=1mm]}-{Latex[length=1mm]}] (1.05,0.2) -- (1.1,0.17);
    \draw [{Latex[length=1mm]}-{Latex[length=1mm]}] (1.09,-0.07) -- (1.05,-0.09);
    \draw [domain=-0.2:1.84] plot(\x,{(-0-0*\x)/1});
    \draw (0,-0.7) -- (0,0.92);
    \begin{scriptsize}
    \fill (0,0) circle (1.5pt);
    \draw (0,0) node[anchor=north east] {$O$};
    \fill (1,0) circle (1.5pt);
    \draw (0.995,-0.00) node[anchor=north] {$T_0$};
    \fill (1.2,0) circle (1.5pt);
    \draw (1.21,0.0) node[anchor=south] {$P$};
    \draw (0,0.4) node[anchor=east] {$Y$};
    \draw (-0.17,-0.63) node {$Y_2$};
    \draw (-0.17,0.63) node {$Y_1$};
    \fill [color=blue] (0.1,0.1) circle (1.5pt);
    \draw (0.1,0.1) node[anchor=south west] {$H$};
    \fill [color=blue] (1,0.1) circle (1.5pt);
    \draw (1.01,0.12) node[anchor=south] {$T$};
    \draw (-0.17,0.72) node {$Y'_1$};
    \draw (-0.17,-0.51) node {$Y'_2$};
    \draw (0.36,0.05) node {$\delta=\delta_0$};
    \draw (0.12,0.26) node {$d=d_0$};
    \draw (0.61,0.28) node {$d_1$};
    \draw (0.71,-0.12) node {$d_2$};
  \end{scriptsize}
  \begin{tiny}
    \draw (1.09,0.21) node {$\delta_{\!1}$};
    \draw (1.1,-0.12) node {$\delta_{\!2}$};
  \end{tiny}
    \end{tikzpicture}
    \caption{Constructions and notation in the proof of 
    Lemma~\ref{lm:branching} and Theorem~\ref{th:main}.}
  \end{center}
\end{figure}
\begin{lemma}\label{lm:branching}
  Let $Y'=\ENCLOSE{x=d}$ for some $d<1$,
  be a line parallel to $Y=\ENCLOSE{x=0}$.
  If $S\in \M(Y'\cup A)$
  and $\lambda < \frac 1 {25}$ then
  there is a branching point $T\in S$ 
  such that $S\setminus T$ is composed of 
  three connected sets the closures of which are
   $[HT]$, $S_1$ and $S_2$ 
  with $H\in Y'$ and $[HT]$ perpendicular to $Y'$.  
  Also: $T\in B_{\lambda^2}(T_0)$ where $T_0=(1,0)$
  and $d(H,\ENCLOSE{y=0})\le \lambda$.
  Finally if $Y_j'$ is the line parallel to $Y_j=f_j(Y)$ 
  passing through $T$ 
  then $S_j \in \M(Y_j'\cup A_j)$ for $j=1,2$.
  Also $Y_j'\subset f_j(\ENCLOSE{x<1})$.
\end{lemma}
\begin{proof}
  We first claim that $S$ cannot contain 
  two compact connected sets 
  $\sigma$ and $\gamma$
  with $\H^1(\sigma\cap \gamma)=0$
  such that $\sigma$ touches both $Y'$ and $A_1$ 
  while $\gamma$ touches both $A_1$ and $A_2$.
  In fact, otherwise
  \begin{equation}
  \label{eq:43747}
  \begin{aligned}
    \H^1(S)
    &\ge \dist(Y', A_1) + \dist(A_1, A_2)
    \\
    &\ge 
    1 - d + \frac{\lambda} 2 
      - \frac{\lambda^2}{2}\frac{1}{1-\lambda^2}
      - \frac{\lambda^3}{2}\frac{1}{1-\lambda^2} 
     +
     \sqrt 3 \lambda - \sqrt 3 \frac{\lambda^3}{1-\lambda}
  \end{aligned}
  \end{equation}
  by Lemma~\ref{lm:precedente1}.
  Let $O'=(d,0)$ and $\Sigma':=[O',T_0] \cup f_1(\Sigma) \cup f_2(\Sigma)$ 
  be the tree obtained by adding or removing a segment of length 
  $\abs{d}$ from $\Sigma$ so that $\Sigma'\in \St(Y'\cup A)$.
  We have  
  \begin{equation}\label{eq:44321}
    \H^1(\Sigma')
    = \H^1(\Sigma) - d 
    = 1 - d + 2 \lambda + 4 \lambda^2 + \dots 
    = \frac{1}{1-2\lambda}-d
  \end{equation}
  and one can check that the rhs of \eqref{eq:43747} is
  strictly greater than the rhs of \eqref{eq:44321}
  for $\lambda < \frac 1 {25}$.
  Hence $\H^1(S) > \H^1(\Sigma')$ contrary 
  to the optimality of $S$.
  
  By \cite{PaoSte} we know that $S$ touches $Y'$ in a single point $H$.
  Consider an arc which connects $H$ to any point of $A$
  and consider the longest (injective) arc $S_0$, along such a curve,
  with endpoints $H\in Y'$ and $T\in S$
  such that $(S\setminus S_0) \cup \ENCLOSE{T}$ is connected.
  
  \emph{Case 1.} 
  Suppose $T$ is a point of $A$.
  Without loss of generality suppose $T\in A_1$.
  The set $(S\setminus S_0)\cup\ENCLOSE{T}$
  is connected and contains points of both $A_1$
  and $A_2$ hence there exists an arc $\gamma$ 
  in $S\setminus \sigma\cup\ENCLOSE{T}$.
  The claim above with $\sigma:=S_0$ implies that this cannot happen.
  
  \emph{Case 2.} Suppose $T\not \in A$.
  In this case $S\setminus S_0$ has two connected components 
  $S_1'$ and $S_2'$ (recall that $S$ contains no loops).
  Let $S_j:=S_j'\cup \ENCLOSE{T}$.
  Each $S_j$ must contain at least one point of $A$ otherwise 
  $S\setminus S_j\in \St(Y\cup A)$ will be 
  strictly shorter than $S\in \M(Y\cup A)$.
  Moreover each point of $A$ is contained in either $S_1$ or $S_2$.
  Without loss of generality suppose that $S_1$ 
  contains at least one point of $A_1$.
  
  \emph{Case 2a.} 
  Suppose that $S_1$ contains also points of $A_2$.
  Then we can apply the claim above with 
  $\sigma:= S_0 \cup S_2$ and 
  $\gamma:= S_1$ and exclude that this case can happen.
  
  \emph{Case 2b.} If $S_2$ contains points of both 
  $A_1$ and $A_2$ we proceed 
  as in Case 2a with $S_1$ and $S_2$ interchanged.
  
  The only remaining possibility is that $S_1$ 
  only touches points of $A_1$ 
  and $S_2$ only touches points of $A_2$. 
  Hence $S_1\supset A_1$ and $S_2\supset A_2$ since $A\subset S$.
  So $S_0\in \M(\ENCLOSE{H,T})$  
  and $S_j \in \M(\ENCLOSE{T} \cup A_j)$ for $j=1,2$
  otherwise, by substituting $S_0$ with an element of $\M(\ENCLOSE{H,T})$
  and $S_j$ with an element of $\M(\ENCLOSE{T}\cup A_j)$,
  we could construct a better competitor 
  to $S\in \St(Y\cup A)$.
  
  \emph{Step 1.} 
  Clearly $S_0\in \M(\ENCLOSE{H,T})$ implies that $S_0$ is the straight 
  line segment $[HT]$ perpendicular to $Y$.
  
  \emph{Step 2.}
  By Lemma~\ref{lm:01} we have $A\subset B_\lambda(P)$,
  hence $A_j = f_j(A) \subset B_{\lambda^2}(f_j(P))$
  for $j=1,2$.
  Since $S_j \in \M(\ENCLOSE{T}\cup A_j)$
  contains a small segment perpendicular to $Y_j$,
  by Lemma~\ref{lm:envelope} applied to $f_j^{-1}(S_j)$
  we obtain that $S_j$ is contained in the strip perpendicular 
  to $Y_1$, centered in $T_0$
  containing $A_j$. 
  Since $A_j\subset B_{\lambda^2}(f_j(P))$
  such a strip has width $\lambda^2$.
  (notice that $f_j(P)$ lies on the line passing through $T_0$ and perpendicular 
  to $Y_j$).
  
  The intersection between the two strips for $j=1,2$ 
  is the union of two equilateral 
  triangles each with height $\lambda^2$. 
  Hence $T$ is contained in the ball
  $B_{\lambda^2}(T_0)$ proving the respective claim of the Lemma.
  Moreover the distance of the line $Y_j'$ from $T_0$ is less than 
  $\lambda^2$ hence $Y_j\subset f_j(\ENCLOSE{x<\lambda})
  \subset f_j(\ENCLOSE{x<1})$.
  
  \emph{Step 3.}
  We claim that each $S_j$, $j=1,2$ has no branching points inside the triangle 
  delimited by $Y$, $Y_1$ and $Y_2$ 
  (to avoid confusion, notice that $T$ is not a branching point of $S_j$).
 
  If $T$ is itself outside of this triangle there is nothing to prove
  because $S_j\subset \overline{\co}(\ENCLOSE{T}\cup A_j)$ 
  is itself outside the triangle.
  Otherwise, 
  since $A_j=f_j(A)\subset B_{\lambda^2}(f_j(P))$
  then by Lemma~\ref{lm:01} all branching points of $S_j$ 
  are inside $B_{\frac{2}{\sqrt 3}\lambda^2}(f_j(P))$ while 
  $d(f_j(P),Y_j)=\lambda d(P,Y)
  = \lambda \enclose{1+\frac{\lambda} 2 }
  > \frac{2}{\sqrt 3}\lambda^2$.
  
  \emph{Step 4: conclusion.}
  We now show the last claim $S_j \in \M(Y_j'\cup A_j)$, $j=1,2$.

  Take any $S_j'\in \M(Y_j'\cup A_j)$ and let $H_j'$ be the 
  endpoint of $S_j'$ on the line $Y_j'$. 
  Clearly $S_j'\in \M(\ENCLOSE{H_j'}\cup A_j)$.
  Let $H_j''$ be the point on $Y_j$ such that $[H_j'' H_j']$ is perpendicular
  to $Y_j$. 
  Clearly $H_j''$ is on the boundary of $\Delta$ since the orthogonal 
  projection of $A_j$ on $Y_j$ is contained in the boundary of $\Delta$.
  By Lemma~\ref{lm:base} we know that $S_j'$ has no branching points
  inside the triangle $\Delta$ delimited by $Y$, $Y_1$ and $Y_2$.
  So we can define a set $S_j''$ which is obtained by adding
  (in case $d_j(T)<0$)
  or removing (in case $d_j(T)>0$)
  the segment $[H_j'' H_j']$ to (resp. from) $S_j'$ 
  so that $S_j''\in \St(Y_j\cup A_j)$.
  In particular $H^1(S_j') = H^1(S_j'') + d_j(T)$,
  where $d_j$ is the signed distance of $T$ from the line $Y_j$ (positive 
  when $T$ is inside $\Delta$).
  Let $\Gamma\in \M(Y\cup\ENCLOSE{H_1'',H_2''})$ 
  be a tripod with endpoints on $H_1''$, $H_2''$ and a 
  point on $Y$. 
  By Lemma~\ref{lm:tripod} $H^1(\Gamma) = 1$.
  And hence, for 
  \[
   S'' := \Gamma \cup S_1'' \cup S_2''  
  \]
  one has 
  \begin{align*}
    \H^1(S'')
    &= 1 + \H^1(S_1'') + \H^1(S_2'') \\
    &= 1 + (\H^1(S_1') - d_1(T)) + (\H^1(S_2') - d_2(T)) \\
    &= (1-d_1(T) - d_2(T)) + \H^1(S_1') + \H^1(S_2') \\
    &= d_3(T) + \H^1(S_1') + \H^1(S_2') \\
    &= \abs{HT} + \H^1(S_1') + \H^1(S_2') \\
    &\le \abs{HT} + \H^1(S_1) + \H^1(S_2)
    = \H^1(S).
  \end{align*}
  Since $S\in \M(Y\cup A)$ while $S''\in \St(Y\cup A)$ the above 
  inequalities are actually equalities.
  In particular the inequalities $\H^1(S_j') \le \H^(S_j)$ are 
  also equalities hence $S_j\in \M(Y_j'\cup A_j)$ as claimed.
\end{proof}

\begin{theorem}\label{th:main}
Let $S\in \M(Y\cup A)$. 
Then $S$ is the standard bifurcation tree.
\end{theorem}
\begin{proof}
Given any $S\in \M(\ENCLOSE{x=d}\cup A)$ with $d<1$, 
by Lemma~\ref{lm:branching} we are able to define $S_1$ and $S_2$, 
$H$, $T$
 such that 
$S=[H T] \cup S_1 \cup S_2$, $[HT]$ is perpendicular to $Y$.
If we define the sets $b_1(S)$ and $b_2(S)$ 
by $b_j(S) := f_j^{-1}(S_j)$, $j=1,2$.
We notice that $b_j(S)\in \M(\ENCLOSE{x=d_j}\cup A)$ for some $d_1<1$, 
$d_2<1$ (as claimed in the Lemma~\ref{lm:branching}).
Let $d(S)$ and $\delta(S)$ be the two coordinates of the point $H$ 
so that $d(S)$ is the distance of $H$ from the line $Y$ 
and $\delta(S)$ is the distance of the point $T$ from the line $X=\ENCLOSE{y=0}$.
Hence Lemma~\ref{lm:branching} can be applied inductively on the two 
rescaled branches $b_1(S)$ and $b_2(S)$.

Define $\S^0:=\ENCLOSE{S}$ where $S\in \M(\ENCLOSE{x=0}\cup A)$ is given
and define inductively $\S^k$ as the family of the rescaled branches of $S$ 
at level $k$:
\[
  \S^{k+1} := \bigcup_{S\in \S^k}\ENCLOSE{b_1(S),b_2(S)}.
\]

We claim that for all $S\in \S^k$, $\delta(S)\le 2^k \lambda^k$.
In fact we know that $S$ touches the vertical line $\ENCLOSE{x=d}$ 
in a single point $H$.
By Lemma~\ref{lm:branching} we know that $S$ is composed 
by a segment $[H T]$ and two trees $S_1$, $S_2$
with $S_j\in \M(Y_j'\cup A_j)$, $j=1,2$
where $Y_j'$ is the line passing through $T$ and parallel to $Y_j$
and we know that $\abs{T T_0} \le \lambda^2$. 
Let $\delta_j$ be the distance of $T$ from $f_j(X)$, $j=1,2$.
Since $T_0\in f_j(X)$ we have $\delta_j\le \abs{T T_0} \le \lambda^2$
(as stated in Lemma~\ref{lm:branching}).
By Lemma~\ref{lm:tripod} applied to the three lines $X$, $f_1(X)$ and $f_3(X)$
we know that the sum of the signed distances of $T$ from the three lines 
passing through $T_0$ is equal to $0$.
Hence
\begin{align*}
  \delta(S) &\le \delta_1 + \delta_2 
    \le 2\max\ENCLOSE{\delta_1,\delta_2}\\
    &= 2\lambda\max\ENCLOSE{\delta(b_1(S)),\delta(b_2(S))}.
\end{align*}
This gives 
\[
  \Delta_k 
  := \max\ENCLOSE{\delta(S)\colon S\in \S^k}
   \le 2\lambda \max\ENCLOSE{\delta(S)\colon S\in \S^{k+1}}
   = 2 \lambda \Delta_{k+1}.
\]
If by contradiction, $\Delta_0>0$ we would have 
$\Delta_k\to +\infty$ while we know,
by Lemma~\ref{lm:branching}, that $\Delta_k \le \lambda$
for all $k$.
\end{proof}
\end{document}