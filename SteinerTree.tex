\documentclass{article}
\usepackage[utf8]{inputenc}

\usepackage{amsmath,amssymb,amsthm}
\usepackage{comment}
\usepackage{hyperref}

\newcommand{\RR}{\mathbb R}
\newcommand{\NN}{\mathbb N}
\newcommand{\ZZ}{\mathbb Z}

\renewcommand{\H}{\mathcal H}
\newcommand{\eps}{\varepsilon}

\newcommand{\loc}{\mathrm{loc}}

\renewcommand{\vec}[1]{\mathbf{#1}}
\newcommand{\abs}[1]{\left\vert #1 \right\vert}
\newcommand{\Abs}[1]{\left\Vert #1 \right\Vert}
\newcommand{\enclose}[1]{\left(#1\right)}
\newcommand{\Enclose}[1]{\left[#1\right]}
\newcommand{\ENCLOSE}[1]{\left\{#1\right\}}
\newcommand{\weakstarto}{{\stackrel{*}{\rightharpoonup}}}

\newcommand{\St}{\mathrm{St}}
\newcommand{\M}{\mathcal{M}}
\renewcommand{\H}{\mathcal{H}}
\newcommand{\dist}{\mathrm{dist}}
\newcommand{\co}{\mathrm{co}}

\newtheorem{theorem}{Theorem}[section]
\newtheorem{proposition}[theorem]{Proposition}
\newtheorem{conjecture}[theorem]{Conjecture}
\newtheorem{lemma}[theorem]{Lemma}
\newtheorem{corollary}[theorem]{Corollary}
\newtheorem{assumption}[theorem]{Assumption}
\newtheorem{problem}[theorem]{Problem}
\newtheorem{openpb}[theorem]{Open Problem}
\theoremstyle{definition}
\newtheorem{definition}[theorem]{Definition}
\theoremstyle{remark}
\newtheorem{remark}[theorem]{Remark}
\newtheorem{example}[theorem]{Example}


\title{Solution to the Steiner problem on an infinite self-similar set}
\author{Paolini Stepanov}

\begin{document}
\maketitle
Let $\St(A)$ be the family of all compact sets $S$ such that 
$S\cup A$ is connected.
Let $\M(A)$ be the the subsets of $\St(A)$ of sets $S$ 
with minimal length $\H^1(S)$.

Let $A$ be the Cantor-like set you know which one: 
$f_j(z) = 1 + \theta z$,
$\theta = \lambda e^{i \frac \pi 3}$.
$A=A_1\cup A_2$ 
with $A_j = f_j(A)$.

Let $Y=\ENCLOSE{(x,y)\colon x=0}$ be the $y$-axis,
let $Y_j = f_j(Y)$.
Consider the point $P=(1+\frac\lambda 2,0)$ 
and the lines $\tilde Y_j$ parallel to $Y_j$ 
passing through $P$.

Let $\Sigma\in \St(Y\cup A)$ be the binary self-similar 
tree you know which one.
We claim that $\ENCLOSE{\Sigma} = \M(Y\cup A)$.

\textbf{Ricordare. Dire che $S$ è connesso. 
Dire che non importa se $Y$ non è limitato.
Dire che dati due punti di $S$ esiste un unico arco che li connette.
Dire cos'è un arco. }

\begin{lemma}\label{lm:base}
  Let $P=(d,0)$, $X$ compact subset of $\overline{B_\rho(P)}$.
  Let $r=2\rho/\sqrt 3$ and suppose that $r<d$.
  Let $S \in \M(Y\cup X)$.
  Then there exists $Q\in \partial B_r(P)$ and $H\in Y$
  such that $S\setminus B_r(P) = [H,Q]$.
  Moreover $[H,Q]$ is perpendicular to $Y$.
\end{lemma}
\begin{proof}
    Otherwise $S$ has a triple point outside $B_r(P)$.
    So the 120 degree angle encloses the whole ball $B_\rho(P)$ 
    and a projection would decrease the length.
\end{proof}
  
\begin{lemma}\label{lm:01}
  Let $P$ and $A$ be defined as above,
  $R=\frac{2\lambda}{\sqrt 3}$.
  Then $A\subset B_\lambda(P)$ and 
  given any $S\in \M(\ENCLOSE{T}\cup A)$ 
  for some $T\not \in B_R(P)$
  we have that $S\setminus B_R(P)$ is a straight line segment.
\end{lemma}

\begin{lemma}\label{lm:tripod}
  Let $Y_1,Y_2,Y_3$ be three lines in $\RR^2$
  forming angles of $60$ degrees so that 
  the three pairwise intersections identify
  an equilateral triangle with sides of length $\ell$.
  Let $T$ be any point and let $d_i(T)$ be 
  the signed distance of $T$ from $Y_i$
  with positive sign when $T$ is inside the triangle.
  Then $d_1(T) + d_2(T) + d_3(T) = \frac{\sqrt 3}{2}\ell$.
\end{lemma}

\begin{lemma}\label{lm:precedente1}
One has
\begin{align*}
    \dist(Y,A_1) & \ge 
    1+ \frac{\lambda} 2 
    - \frac{\lambda^2(1+2\lambda)}{2(1-\lambda^2)}\\
    \dist(A_1,A_2) & 
    \ge 
    \sqrt 3 \lambda - \sqrt 3 \frac{\lambda^3}{1-\lambda}.
\end{align*}
\end{lemma}
\begin{proof}
  \begin{align*}
  \dist(Y,A_1) &\ge 1+ \frac \lambda 2 
     - \enclose{\frac{\lambda^2}{2} 
      + \lambda^3 
      + \frac{\lambda^4 }{2}
      + \lambda^5
      + \frac{\lambda^6}{4} + \dots}\\
      &= 1 + \frac \lambda 2 
      - \frac{1}{2}\frac{\lambda^2}{1-\lambda^2}
      - \frac{\lambda^3}{1-\lambda^2}\\
      &= 1+ \frac{\lambda} 2 
      - \frac{\lambda^2(1+2\lambda)}{2(1-\lambda^2)}
  \end{align*}
  while 
  \begin{align*}
    \dist(A_1,A_2)
    &\ge 2\enclose{\frac{\sqrt 3}{2} \lambda 
      -\frac{\sqrt 3}{2}\enclose{\lambda^3 + \lambda^4 + \dots}
     }\\
    &= \sqrt 3 \lambda -\sqrt 3 \frac{\lambda^3}{1-\lambda}.
  \end{align*}
\end{proof}

\begin{lemma}\label{lm:envelope}
Let $\Sigma\in \M(\ENCLOSE{T}\cup X)$
for some compact set $X$ contained in a horizontal strip 
$E=\ENCLOSE{(x,y)\colon \abs{y}\le d}$ of width $2d>0$.
Suppose also that there is some $T'\in \Sigma$, $T'\neq T$ 
such that $[T,T']$ is horizontal.
Then $T\subset E$.
\end{lemma}
%
\begin{proof}
  Suppose by contradiction that $T=(x_T,y_T)$ is outside $E$.
  For example $y_T>d$. 
  Then the convex hull of $\ENCLOSE{T}\cup X$ has a single 
  point, which is $T$, on the line $\ENCLOSE{y=y_T}$. 
  This is in contradiction with the fact that $T'\in \Sigma$ 
  has the same $y$-coordinate as $T$.
\end{proof}

\begin{lemma}\label{lm:branching}
  Let $Y'=\ENCLOSE{x=d}$ for some $d<1$,
  be a line parallel to $Y=\ENCLOSE{x=0}$.
  If $S\in \M(Y'\cup A)$
  and $\lambda < \frac 1 {25}$ then
  there is a branching point $T\in S$ 
  such that $S\setminus T$ is composed of 
  three connected sets the closures of which are
   $[HT]$, $S_1$ and $S_2$ 
  with $H\in Y'$ and $[HT]$ perpendicular to $Y'$.  
  Also: $T\in B_{\lambda^2}(T_0)$ where $T_0=(1,0)$.
  Finally if $Y_j'$ is the line parallel to $Y_j=f_j(Y)$ 
  passing through $T$ 
  then $S_j \in \M(Y_j'\cup A_j)$ for $j=1,2$.
  Also $Y_j'\subset f_j(\ENCLOSE{x<1})$.
\end{lemma}
\begin{proof}
  We first claim that $S$ cannot contain 
  two compact connected sets 
  $\sigma$ and $\gamma$
  with $\H^1(\sigma\cap \gamma)=0$
  such that $\sigma$ touches both $Y'$ and $A_1$ 
  while $\gamma$ touches both $A_1$ and $A_2$.
  In fact, otherwise
  \begin{equation}
  \label{eq:43747}
  \begin{aligned}
    \H^1(S)
    &\ge \dist(Y', A_1) + \dist(A_1, A_2)
    \\
    &\ge 
    1 - d + \frac{\lambda} 2 
      - \frac{\lambda^2}{2}\frac{1}{1-\lambda^2}
      - \frac{\lambda^3}{2}\frac{1}{1-\lambda^2} 
     +
     \sqrt 3 \lambda - \sqrt 3 \frac{\lambda^3}{1-\lambda}
  \end{aligned}
  \end{equation}
  by Lemma~\ref{lm:precedente1}.
  Let $O'=(d,0)$ and $\Sigma':=[O',T_0] \cup f_1(\Sigma) \cup f_2(\Sigma)$ 
  be the tree obtained by adding or removing a segment of length 
  $\abs{d}$ from $\Sigma$ so that $\Sigma'\in \St(Y'\cup A)$.
  We have  
  \begin{equation}\label{eq:44321}
    \H^1(\Sigma')
    = \H^1(\Sigma) - d 
    = 1 - d + 2 \lambda + 4 \lambda^2 + \dots 
    = \frac{1}{1-2\lambda}-d
  \end{equation}
  and one can check that the rhs of \eqref{eq:43747} is
  strictly greater than the rhs of \eqref{eq:44321}
  for $\lambda < \frac 1 {25}$.
  Hence $\H^1(S) > \H^1(\Sigma')$ contrary 
  to the optimality of $S$.
  
  By \cite{PaoSte} we know that $S$ touches $Y'$ in a single point $H$.
  Consider an arc which connects $H$ to any point of $A$
  and consider the longest (injective) arc $S_0$, along such a curve,
  with endpoints $H\in Y'$ and $T\in S$
  such that $(S\setminus S_0) \cup \ENCLOSE{T}$ is connected.
  
  \emph{Case 1.} 
  Suppose $T$ is a point of $A$.
  Without loss of generality suppose $T\in A_1$.
  The set $(S\setminus S_0)\cup\ENCLOSE{T}$
  is connected and contains points of both $A_1$
  and $A_2$ hence there exists an arc $\gamma$ 
  in $S\setminus \sigma\cup\ENCLOSE{T}$.
  The claim above with $\sigma:=S_0$ implies that this cannot happen.
  
  \emph{Case 2.} Suppose $T\not \in A$.
  In this case $S\setminus S_0$ has two connected components 
  $S_1'$ and $S_2'$ (recall that $S$ contains no loops).
  Let $S_j:=S_j'\cup \ENCLOSE{T}$.
  Each $S_j$ must contain at least one point of $A$ otherwise 
  $S\setminus S_j\in \St(Y\cup A)$ will be 
  strictly shorter than $S\in \M(Y\cup A)$.
  Moreover each point of $A$ is contained in either $S_1$ or $S_2$.
  Without loss of generality suppose that $S_1$ 
  contains at least one point of $A_1$.
  
  \emph{Case 2a.} 
  Suppose that $S_1$ contains also points of $A_2$.
  Then we can apply the claim above with 
  $\sigma:= S_0 \cup S_2$ and 
  $\gamma:= S_1$ and exclude that this case can happen.
  
  \emph{Case 2b.} If $S_2$ contains points of both 
  $A_1$ and $A_2$ we proceed 
  as in Case 2a with $S_1$ and $S_2$ interchanged.
  
  The only remaining possibility is that $S_1$ 
  only touches points of $A_1$ 
  and $S_2$ only touches points of $A_2$. 
  Hence $S_1\supset A_1$ and $S_2\supset A_2$ since $A\subset S$.
  So $S_0\in \M(\ENCLOSE{H,T})$  
  and $S_j \in \M(\ENCLOSE{T} \cup A_j)$ for $j=1,2$
  otherwise, by substituting $S_0$ with an element of $\M(\ENCLOSE{H,T})$
  and $S_j$ with an element of $\M(\ENCLOSE{T}\cup A_j)$,
  we could construct a better competitor 
  to $S\in \St(Y\cup A)$.
  
  \emph{Step 1.} 
  Clearly $S_0\in \M(\ENCLOSE{H,T})$ implies that $S_0$ is the straight 
  line segment $[HT]$ perpendicular to $Y$.
  
  \emph{Step 2.}
  By Lemma~\ref{lm:01} we have $A\subset B_\lambda(P)$,
  hence $A_j = f_j(A) \subset B_{\lambda^2}(f_j(P))$
  for $j=1,2$.
  Since $S_j \in \M(\ENCLOSE{T}\cup A_j)$
  contains a small segment perpendicular to $Y_j$,
  by Lemma~\ref{lm:envelope} applied to $f_j^{-1}(S_j)$
  we obtain that $S_j$ is contained in the strip perpendicular 
  to $Y_1$, centered in $T_0$
  containing $A_j$. 
  Since $A_j\subset B_{\lambda^2}(f_j(P))$
  such a strip has width $\lambda^2$.
  (notice that $f_j(P)$ lies on the line passing through $T_0$ and perpendicular 
  to $Y_j$).
  
  The intersection between the two strips for $j=1,2$ 
  is the union of two equilateral 
  triangles each with height $\lambda^2$. 
  Hence $T$ is contained in the ball
  $B_{\lambda^2}(T_0)$ proving the respective claim of the Lemma.
  Moreover the distance of the line $Y_j'$ from $T_0$ is less than 
  $\lambda^2$ hence $Y_j\subset f_j(\ENCLOSE{x<\lambda})
  \subset f_j(\ENCLOSE{x<1})$.
  
  \emph{Step 3.}
  We claim that each $S_j$, $j=1,2$ has no branching points inside the triangle 
  delimited by $Y$, $Y_1$ and $Y_2$ 
  (to avoid confusion, notice that $T$ is not a branching point of $S_j$).
 
  If $T$ is itself outside of this triangle there is nothing to prove
  because $S_j\subset \overline{\co}(\ENCLOSE{T}\cup A_j)$ 
  is itself outside the triangle.
  Otherwise, 
  since $A_j=f_j(A)\subset B_{\lambda^2}(f_j(P))$
  then by Lemma~\ref{lm:01} all branching points of $S_j$ 
  are inside $B_{\frac{2}{\sqrt 3}\lambda^2}(f_j(P))$ while 
  $d(f_j(P),Y_j)=\lambda d(P,Y)
  = \lambda \enclose{1+\frac{\lambda} 2 }
  > \frac{2}{\sqrt 3}\lambda^2$.
  
  \emph{Step 4: conclusion.}
  We now show the last claim $S_j \in \M(Y_j'\cup A_j)$, $j=1,2$.

  Take any $S_j'\in \M(Y_j'\cup A_j)$ and let $H_j'$ be the 
  endpoint of $S_j'$ on the line $Y_j'$. 
  Clearly $S_j'\in \M(\ENCLOSE{H_j'}\cup A_j)$.
  Let $H_j''$ be the point on $Y_j$ such that $[H_j'' H_j']$ is perpendicular
  to $Y_j$. 
  Clearly $H_j''$ is on the boundary of $\Delta$ since the orthogonal 
  projection of $A_j$ on $Y_j$ is contained in the boundary of $\Delta$.
  By Lemma~\ref{lm:base} we know that $S_j'$ has no branching points
  inside the triangle $\Delta$ delimited by $Y$, $Y_1$ and $Y_2$.
  So we can define a set $S_j''$ which is obtained by adding or removing 
  the segment $[H_j'' H_j']$ from $S_j'$ so that $S_j''\in \St(Y_j\cup A_j)$.
  In particular $H^1(S_j') = H^1(S_j'') + d_j(T)$,
  where $d_j$ is the signed distance of $T$ from the line $Y_j$ (positive 
  when $T$ is inside $\Delta$).
  Let $\Gamma\in \M(Y\cup\ENCLOSE{H_1'',H_2''})$ 
  be a tripod with endpoints on $H_1''$, $H_2''$ and a 
  point on $Y$. 
  By Lemma~\ref{lm:tripod} $H^1(\Gamma) = 1$.
  And hence, for 
  \[
   S'' := \Gamma \cup S_1'' \cup S_2''  
  \]
  one has 
  \begin{align*}
    \H^1(S'')
    &= 1 + \H^1(S_1'') + \H^1(S_2'') \\
    &= 1 + (\H^1(S_1') - d_1(T)) + (\H^1(S_2') - d_2(T)) \\
    &= (1-d_1(T) - d_2(T)) + \H^1(S_1') + \H^1(S_2') \\
    &= d_3(T) + \H^1(S_1') + \H^1(S_2') \\
    &= \abs{HT} + \H^1(S_1') + \H^1(S_2') \\
    &\le \abs{HT} + \H^1(S_1) + \H^1(S_2)
    = \H^1(S).
  \end{align*}
  Since $S\in \M(Y\cup A)$ while $S''\in \St(Y\cup A)$ the above 
  inequalities are actually equalities.
  In particular the inequalities $\H^1(S_j') \le \H^(S_j)$ are 
  also equalities hence $S_j\in \M(Y_j'\cup A_j)$ as claimed.
\end{proof}

\begin{theorem}\label{th:main}
Let $S\in \M(Y\cup A)$. 
Then $S$ is the standard bifurcation tree.
\end{theorem}
\begin{proof}
We know that $S$ touches $Y$ in a single point $H$.

\end{proof}

\end{document}