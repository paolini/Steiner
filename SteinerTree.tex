\documentclass{article}
\usepackage[utf8]{inputenc}

\usepackage{amsmath,amssymb,amsthm}
\usepackage{comment}
\usepackage{hyperref}

\newcommand{\RR}{\mathbb R}
\newcommand{\NN}{\mathbb N}
\newcommand{\ZZ}{\mathbb Z}

\renewcommand{\H}{\mathcal H}
\newcommand{\eps}{\varepsilon}

\newcommand{\loc}{\mathrm{loc}}

\renewcommand{\vec}[1]{\mathbf{#1}}
\newcommand{\abs}[1]{\left\vert #1 \right\vert}
\newcommand{\Abs}[1]{\left\Vert #1 \right\Vert}
\newcommand{\enclose}[1]{\left(#1\right)}
\newcommand{\Enclose}[1]{\left[#1\right]}
\newcommand{\ENCLOSE}[1]{\left\{#1\right\}}
\newcommand{\weakstarto}{{\stackrel{*}{\rightharpoonup}}}

\newcommand{\St}{\mathrm{St}}
\newcommand{\M}{\mathcal{M}}
\renewcommand{\H}{\mathcal{H}}
\newcommand{\dist}{\mathrm{dist}}

\newtheorem{theorem}{Theorem}[section]
\newtheorem{proposition}[theorem]{Proposition}
\newtheorem{conjecture}[theorem]{Conjecture}
\newtheorem{lemma}[theorem]{Lemma}
\newtheorem{corollary}[theorem]{Corollary}
\newtheorem{assumption}[theorem]{Assumption}
\newtheorem{problem}[theorem]{Problem}
\newtheorem{openpb}[theorem]{Open Problem}
\theoremstyle{definition}
\newtheorem{definition}[theorem]{Definition}
\theoremstyle{remark}
\newtheorem{remark}[theorem]{Remark}
\newtheorem{example}[theorem]{Example}


\title{Solution to the Steiner problem on an infinite self-similar set}
\author{Paolini Stepanov}

\begin{document}
\maketitle
Let $\St(A)$ be the family of all compact sets $S$ such that 
$S\cup A$ is connected.
Let $\M(A)$ be the the subsets of $\St(A)$ of sets $S$ 
with minimal length $\H^1(S)$.

Let $A$ be the Cantor-like set you know which one: $A=A_1\cup A_2$ 
with $A_1 = 1+\theta A$, $A_2 = 1+\bar \theta A$,
$\theta = \lambda e^{i \frac \pi 3}$.
Let $W=[-ih,ih]$ where $h=\frac{\sqrt{3}} 2 \lambda$.

Let $Y=\ENCLOSE{(x,y)\colon x=0}$ be the $y$-axis.

Let $\Sigma\in \St(Y\cup A)$ be the binary self-similar 
tree you know which one.
We claim that $\ENCLOSE{\Sigma} = \M(Y\cup A)$.

\textbf{Ricordare. Dire che $S$ è connesso. 
Dire che non importa se $Y$ non è limitato.
Dire che dati due punti di $S$ esiste un unico arco che li connette.
Dire cos'è un arco. }

\begin{lemma}
Let $P=(d,0)$, $X$ compact subset of $\overline{B_\rho(P)}$.
Let $r=2\rho/\sqrt 3$ and suppose that $r<d$.
Let $S \in \M(Y\cup X)$.
Then there exists $Q\in \partial B_r(P)$ and $H\in Y$
such that $S\setminus B_r(P) = [H,Q]$.
Moreover $[H,Q]$ is perpendicular to $Y$.
\end{lemma}
\begin{proof}
    Otherwise $S$ has a triple point outside $B_r(P)$.
    So the 120 degree angle encloses the whole ball $B_\rho(P)$ 
    and a projection would decrease the length.
\end{proof}

\begin{lemma}\label{lm:precedente1}
One has
\begin{align*}
    \dist(Y,A_1) & \ge 
    1+ \frac{\lambda} 2 
    - \frac{\lambda^2(1+2\lambda)}{2(1-\lambda^2)}\\
    \dist(A_1,A_2) & 
    \ge 
    \sqrt 3 \lambda - \sqrt 3 \frac{\lambda^3}{1-\lambda}.
\end{align*}
\end{lemma}
\begin{proof}
  \begin{align*}
  \dist(Y,A_1) &\ge 1+ \frac \lambda 2 
     - \enclose{\frac{\lambda^2}{2} 
      + \lambda^3 
      + \frac{\lambda^4 }{2}
      + \lambda^5
      + \frac{\lambda^6}{4} + \dots}\\
      &= 1 + \frac \lambda 2 
      - \frac{1}{2}\frac{\lambda^2}{1-\lambda^2}
      - \frac{\lambda^3}{1-\lambda^2}\\
      &= 1+ \frac{\lambda} 2 
      - \frac{\lambda^2(1+2\lambda)}{2(1-\lambda^2)}
  \end{align*}
  while 
  \begin{align*}
    \dist(A_1,A_2)
    &\ge 2\enclose{\frac{\sqrt 3}{2} \lambda 
      -\frac{\sqrt 3}{2}\enclose{\lambda^3 + \lambda^4 + \dots}
     }\\
    &= \sqrt 3 \lambda -\sqrt 3 \frac{\lambda^3}{1-\lambda}.
  \end{align*}
\end{proof}

\begin{lemma}\label{lm:branching}
Let $S\in \M(Y\cup A)$
and $\lambda < \frac 1 {25}$ then $S$ 
there is a branching point $T\in S$ 
such that $S\setminus T$ is composed of 
three connected sets $S_0$, $S_1$ and $S_2$ 
where $S_0 = [H,T]$ with $H\in Y$ and  
$S_i \in \M(T\cup A_i)$ for $i=1,2$.
Moreover $S_0$ is perpendicular to $Y$.
\end{lemma}
\begin{proof}
We first claim that $S$ cannot contain 
two compact connected sets 
$\sigma$ and $\gamma$
with $\H^1(\sigma\cap \gamma)=0$
such that $\sigma$ touches both $Y$ and $A_1$ 
while $\gamma$ touches both $A_1$ and $A_2$.
In fact, otherwise
\begin{equation}
\label{eq:43747}
\begin{aligned}
  \H^1(S)
  &\ge \dist(Y, A_1) + \dist(A_1, A_2)
  \\
  &\ge 
  1+ \frac{\lambda} 2 
    - \frac{\lambda^2}{2}\frac{1}{1-\lambda^2}
    - \frac{\lambda^3}{2}\frac{1}{1-\lambda^2} 
   +
   \sqrt 3 \lambda - \sqrt 3 \frac{\lambda^3}{1-\lambda}
\end{aligned}
\end{equation}
by Lemma~\ref{lm:precedente1}.
But 
\begin{equation}\label{eq:44321}
  \H^1(\Sigma) 
  = 1 + 2 \lambda + 4 \lambda^2 + \dots 
  = \frac{1}{1-2\lambda}
\end{equation}
and one can check that the rhs of \eqref{eq:43747} is
strictly greater than the rhs of \eqref{eq:44321}
for $\lambda < \frac 1 {25}$,
so that $\H^1(S) > \H^1(\Sigma)$ contrary 
to the optimality of $S$.

By \cite{PaoSte} we know that $S$ touches $Y$ in a single point $H$.
Consider an arc which connects $H$ to any point of $A$ 
and consider the longest arc $S_0$, along such a curve, 
with endpoints $H\in Y$ and $T\in S$
such that $(S\setminus S_0) \cup \ENCLOSE{T}$ is connected.

\emph{Case 1.} Suppose $T$ is a point of $A$. 
Without loss of generality suppose $T\in A_1$.
The set $(S\setminus S_0)\cup\ENCLOSE{T}$ 
is connected and contains points of both $A_1$ 
and $A_2$ hence there exists an arc $\gamma$ in $S\setminus \sigma\cup\ENCLOSE{T}$.
The claim above with $\sigma:=S_0$ implies that this cannot happen.

\emph{Case 2.} Suppose $T\not \in A$.
In this case $S\setminus S_0$ has two connected components 
$S_1'$ and $S_2'$ (recall that $S$ contains no loops). 
Let $S_i:=S_i'\cup \ENCLOSE{T}$.
Each $S_i$ must contain at least one point of $A$ otherwise 
$S\setminus S_i\in \St(Y\cup A)$ will be 
strictly shorter than $S\in \M(Y\cup A)$.
Moreover each point of $A$ is contained in either $S_1$ or $S_2$.
Without loss of generality suppose that $S_1$ 
contains at least one point of $A_1$.

\emph{Case 2a.} Suppose that $S_1$ contains also points of $A_2$.
Then we can apply the claim above with 
$\sigma:= S_0 \cup S_2$ and 
$\gamma:= S_1$ and exclude that this case can happen.

\emph{Case 2b.} If $S_2$ contains points of both $A_1$ and $A_2$ we proceed 
as in Case 2a with $S_1$ and $S_2$ interchanged.

The only remaining possibility is that $S_1$ only touches points of $A_1$ 
and $S_2$ only touches points of $A_2$. 
Hence $S_1\supset A_1$ and $S_2\supset A_2$ since $A\subset S$.
So $S_0\in \M(\ENCLOSE{H,T})$  
and $S_i \in \M(\ENCLOSE{T} \cup A_i)$ for $i=1,2$
otherwise we could construct a better competitor \textbf{[dire meglio]} to $S\in \St(Y\cup A)$.

Clearly $S_0\in \M(\Enclose{H,T})$ implies that $S_0$ is the straight 
line segment $[H,T]$ perpendicular to $Y$.
\end{proof}

\begin{lemma}
  Let $S\in \M(Y\cup A)$ and $\lambda<\frac 1 {25}$.
  Let $S_0$, $S_1$ and $S_2$ be given as in the statement 
  of Lemma~\ref{lemma}
  
\end{lemma}

\begin{lemma}\label{lm:iteration}
  Let $S\in\M(Y\cup A)$ and suppose $S_0=[H,T]$, $S_1$ and $S_2$
  be as in Lemma~\ref{lm:branching}.
  Let $Y_1$ (and $Y_2$) be the line(s) obtained by rotating 
  $Y$ by $\pi/3$ counter-clockwise (clockwise)
  and passing through $T$. 
\end{lemma}

\begin{lemma}
  Let $S\in \M(Y\cup A)$ and let $H$ be the single point 
  in $H\cap 
\end{lemma}

\begin{lemma}
  If $S\in \M(Y\cup A)$ then $S\cap Y$ is composed by a single 
  point $H$ which lies on the symmetry axis of $A$.
\end{lemma}
\begin{proof}

\end{proof}

\end{document}