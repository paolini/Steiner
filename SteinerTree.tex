\documentclass{article}
\usepackage[utf8]{inputenc}

\usepackage{amsmath,amssymb,amsthm}
\usepackage{comment}
\usepackage{hyperref}

\newcommand{\RR}{\mathbb R}
\newcommand{\NN}{\mathbb N}
\newcommand{\ZZ}{\mathbb Z}

\renewcommand{\H}{\mathcal H}
\newcommand{\eps}{\varepsilon}

\newcommand{\loc}{\mathrm{loc}}

\renewcommand{\vec}[1]{\mathbf{#1}}
\newcommand{\abs}[1]{\left\vert #1 \right\vert}
\newcommand{\Abs}[1]{\left\Vert #1 \right\Vert}
\newcommand{\enclose}[1]{\left(#1\right)}
\newcommand{\Enclose}[1]{\left[#1\right]}
\newcommand{\ENCLOSE}[1]{\left\{#1\right\}}
\newcommand{\weakstarto}{{\stackrel{*}{\rightharpoonup}}}

\newcommand{\St}{\mathrm{St}}
\newcommand{\M}{\mathcal{M}}
\renewcommand{\H}{\mathcal{H}}
\newcommand{\dist}{\mathrm{dist}}

\newtheorem{theorem}{Theorem}[section]
\newtheorem{proposition}[theorem]{Proposition}
\newtheorem{conjecture}[theorem]{Conjecture}
\newtheorem{lemma}[theorem]{Lemma}
\newtheorem{corollary}[theorem]{Corollary}
\newtheorem{assumption}[theorem]{Assumption}
\newtheorem{problem}[theorem]{Problem}
\newtheorem{openpb}[theorem]{Open Problem}
\theoremstyle{definition}
\newtheorem{definition}[theorem]{Definition}
\theoremstyle{remark}
\newtheorem{remark}[theorem]{Remark}
\newtheorem{example}[theorem]{Example}


\title{Solution to the Steiner problem on an infinite self-similar set}
\author{Paolini Stepanov}

\begin{document}
\maketitle
Let $\St(A)$ be the family of all compact sets $S$ such that 
$S\cup A$ is connected.
Let $\M(A)$ be the the subsets of $\St(A)$ of sets $S$ 
with minimal length $\H^1(S)$.

Let $A$ be the Cantor-like set you know which one: 
$f_j(z) = 1 + \theta z$,
$\theta = \lambda e^{i \frac \pi 3}$.
$A=A_1\cup A_2$ 
with $A_j = f_j(A)$.

Let $Y=\ENCLOSE{(x,y)\colon x=0}$ be the $y$-axis,
let $Y_j = f_j(Y)$.
Consider the point $P=(1+\frac\lambda 2,0)$ 
and the lines $\tilde Y_j$ parallel to $Y_j$ 
passing through $P$.

Let $\Sigma\in \St(Y\cup A)$ be the binary self-similar 
tree you know which one.
We claim that $\ENCLOSE{\Sigma} = \M(Y\cup A)$.

\textbf{Ricordare. Dire che $S$ è connesso. 
Dire che non importa se $Y$ non è limitato.
Dire che dati due punti di $S$ esiste un unico arco che li connette.
Dire cos'è un arco. }

\begin{lemma}\label{lm:01}
  Let $P$ and $A$ be defined as above,
  $R=\frac{2\lambda}{\sqrt 3}$.
  Then $A\subset B_\lambda(P)$ and 
  given any $S\in \M(\ENCLOSE{T}\cup A)$ 
  for some $T\not \in B_R(P)$
  we have that $S\setminus B_R(P)$ is a straight line segment.
\end{lemma}

\begin{lemma}\label{lm:tripod}
  Let $Y_1,Y_2,Y_3$ be three lines in $\RR^2$
  forming angles of $60$ degrees so that 
  the three pairwise intersections identify
  an equilateral triangle with sides of length $\ell$.
  Let $T$ be any point and let $d_i(T)$ be 
  the signed distance of $T$ from $Y_i$
  with positive sign when $T$ is inside the triangle.
  Then $d_1(T) + d_2(T) + d_3(T) = \frac{\sqrt 3}{2}\ell$.
\end{lemma}

\begin{lemma}
Let $P=(d,0)$, $X$ compact subset of $\overline{B_\rho(P)}$.
Let $r=2\rho/\sqrt 3$ and suppose that $r<d$.
Let $S \in \M(Y\cup X)$.
Then there exists $Q\in \partial B_r(P)$ and $H\in Y$
such that $S\setminus B_r(P) = [H,Q]$.
Moreover $[H,Q]$ is perpendicular to $Y$.
\end{lemma}
\begin{proof}
    Otherwise $S$ has a triple point outside $B_r(P)$.
    So the 120 degree angle encloses the whole ball $B_\rho(P)$ 
    and a projection would decrease the length.
\end{proof}

\begin{lemma}\label{lm:precedente1}
One has
\begin{align*}
    \dist(Y,A_1) & \ge 
    1+ \frac{\lambda} 2 
    - \frac{\lambda^2(1+2\lambda)}{2(1-\lambda^2)}\\
    \dist(A_1,A_2) & 
    \ge 
    \sqrt 3 \lambda - \sqrt 3 \frac{\lambda^3}{1-\lambda}.
\end{align*}
\end{lemma}
\begin{proof}
  \begin{align*}
  \dist(Y,A_1) &\ge 1+ \frac \lambda 2 
     - \enclose{\frac{\lambda^2}{2} 
      + \lambda^3 
      + \frac{\lambda^4 }{2}
      + \lambda^5
      + \frac{\lambda^6}{4} + \dots}\\
      &= 1 + \frac \lambda 2 
      - \frac{1}{2}\frac{\lambda^2}{1-\lambda^2}
      - \frac{\lambda^3}{1-\lambda^2}\\
      &= 1+ \frac{\lambda} 2 
      - \frac{\lambda^2(1+2\lambda)}{2(1-\lambda^2)}
  \end{align*}
  while 
  \begin{align*}
    \dist(A_1,A_2)
    &\ge 2\enclose{\frac{\sqrt 3}{2} \lambda 
      -\frac{\sqrt 3}{2}\enclose{\lambda^3 + \lambda^4 + \dots}
     }\\
    &= \sqrt 3 \lambda -\sqrt 3 \frac{\lambda^3}{1-\lambda}.
  \end{align*}
\end{proof}

\begin{lemma}\label{lm:branching}
Let $S\in \M(Y\cup A)$
and $\lambda < \frac 1 {25}$ then
there is a branching point $T\in S$ 
such that $S\setminus T$ is composed of 
three connected sets the closures of which are
 $[HT]$, $\tilde S_1$ and $\tilde S_2$ 
with $H\in Y$ and  
$\tilde S_j \in \M(T\cup A_j)$ for $j=1,2$.
Moreover
\begin{enumerate}
  \item
 $[HT]$ is perpendicular to $Y$.
\item
$T\in B_{\lambda^2}(T_0)$
where $T_0=(1,0)$
and 
\item $\tilde S_j$, inside the triangle delimited by the lines 
$Y,\tilde Y_1,\tilde Y_2$ (see notation) is a straight line segment $\sigma_j$ 
perpendicular to 
$Y_j$ (or $\tilde Y_j$ which is parallel to $Y_j$).
\end{enumerate}
Construct $S_j$ by adding or removing from $\tilde S_j$ 
a suitable line segment so that $S_j$ has an endpoint in $H_j$ 
which is the intersection of $Y_j$ with the line containing 
$\sigma_j$.
Then $S_j \in \M(Y_j\cup A_j)$.
\end{lemma}
\begin{proof}
We first claim that $S$ cannot contain 
two compact connected sets 
$\sigma$ and $\gamma$
with $\H^1(\sigma\cap \gamma)=0$
such that $\sigma$ touches both $Y$ and $A_1$ 
while $\gamma$ touches both $A_1$ and $A_2$.
In fact, otherwise
\begin{equation}
\label{eq:43747}
\begin{aligned}
  \H^1(S)
  &\ge \dist(Y, A_1) + \dist(A_1, A_2)
  \\
  &\ge 
  1+ \frac{\lambda} 2 
    - \frac{\lambda^2}{2}\frac{1}{1-\lambda^2}
    - \frac{\lambda^3}{2}\frac{1}{1-\lambda^2} 
   +
   \sqrt 3 \lambda - \sqrt 3 \frac{\lambda^3}{1-\lambda}
\end{aligned}
\end{equation}
by Lemma~\ref{lm:precedente1}.
But 
\begin{equation}\label{eq:44321}
  \H^1(\Sigma) 
  = 1 + 2 \lambda + 4 \lambda^2 + \dots 
  = \frac{1}{1-2\lambda}
\end{equation}
and one can check that the rhs of \eqref{eq:43747} is
strictly greater than the rhs of \eqref{eq:44321}
for $\lambda < \frac 1 {25}$,
so that $\H^1(S) > \H^1(\Sigma)$ contrary 
to the optimality of $S$.

By \cite{PaoSte} we know that $S$ touches $Y$ in a single point $H$.
Consider an arc which connects $H$ to any point of $A$
and consider the longest (injective) arc $S_0$, along such a curve,
with endpoints $H\in Y$ and $T\in S$
such that $(S\setminus S_0) \cup \ENCLOSE{T}$ is connected.

\emph{Case 1.} Suppose $T$ is a point of $A$.
Without loss of generality suppose $T\in A_1$.
The set $(S\setminus S_0)\cup\ENCLOSE{T}$
is connected and contains points of both $A_1$
and $A_2$ hence there exists an arc $\gamma$ in $S\setminus \sigma\cup\ENCLOSE{T}$.
The claim above with $\sigma:=S_0$ implies that this cannot happen.

\emph{Case 2.} Suppose $T\not \in A$.
In this case $S\setminus S_0$ has two connected components 
$S_1'$ and $S_2'$ (recall that $S$ contains no loops). 
Let $\tilde S_j:=S_j'\cup \ENCLOSE{T}$.
Each $\tilde S_j$ must contain at least one point of $A$ otherwise 
$S\setminus \tilde S_j\in \St(Y\cup A)$ will be 
strictly shorter than $S\in \M(Y\cup A)$.
Moreover each point of $A$ is contained in either $\tilde S_1$ or $\tilde S_2$.
Without loss of generality suppose that $\tilde S_1$ 
contains at least one point of $A_1$.

\emph{Case 2a.} Suppose that $\tilde S_1$ contains also points of $A_2$.
Then we can apply the claim above with 
$\sigma:= S_0 \cup \tilde S_2$ and 
$\gamma:= \tilde S_1$ and exclude that this case can happen.

\emph{Case 2b.} If $\tilde S_2$ contains points of both 
$A_1$ and $A_2$ we proceed 
as in Case 2a with $\tilde S_1$ and $\tilde S_2$ interchanged.

The only remaining possibility is that $\tilde S_1$ 
only touches points of $A_1$ 
and $\tilde S_2$ only touches points of $A_2$. 
Hence $\tilde S_1\supset A_1$ and $\tilde S_2\supset A_2$ since $A\subset S$.
So $S_0\in \M(\ENCLOSE{H,T})$  
and $S_j \in \M(\ENCLOSE{T} \cup A_j)$ for $j=1,2$
otherwise we could construct a better competitor 
\textbf{[dire meglio]} to $S\in \St(Y\cup A)$.

\emph{Step 1.} 
Clearly $S_0\in \M(\Enclose{H,T})$ implies that $S_0$ is the straight 
line segment $[HT]$ perpendicular to $Y$.

\emph{Step 2.}
By Lemma~\ref{lm:01} we have $A\subset B_\lambda(P)$,
hence $A_j = f_j(A) \subset B_{\lambda^2}(f_j(P))$
for $j=1,2$.
Since $\tilde S_j \in \M(\ENCLOSE{T}\cup A_j)$, $\tilde S_j$ is contained 
in the convex envelope of $\ENCLOSE{T}\cup A_1$, in particular $T$ 
is contained in the strip perpendicular to $Y_1$, centered in $T_0$
containing $B_{\lambda^2}(f_j(P))$ with width $\lambda^2$ 
(notice that $f_j(P)$ lies on the line passing through $T_0$ and perpendicular 
to $Y_j$).

The intersection between the two strips is the union of two equilateral 
triangles each with height $\lambda^2$. 
Hence $T$ is contained in the ball 
$B_{\lambda^2}(T_0)$ proving claim 2 of the lemma.

\emph{Step 3.} 
By Lemma~\ref{lm:01} we have that $A_j=f_j(A)\subset B_{\lambda^2}(f_j(P))$
and $\tilde S_j \in \M(\ENCLOSE{T}\cup A_j)$
hence $\tilde S_j \setminus B_{\lambda R}(f_j(P))$ is a
straight line segment.
To prove claim 3 of the lemma it is enough to note that 
$B_{\lambda R}(f_j(P)) = B_{\frac{2}{\sqrt 3}\lambda^2}(f_j(P))$
is outside of the triangle formed by the lines $Y$, $\tilde Y_1$ 
and $\tilde Y_2$ if $\lambda\le \frac{1}{25}$.

\emph{Step 4: conclusion.}
Let $S_j$, $j=1,2$ be as announced in the statement.
Notice that inside the triangle delimited by the lines 
$Y$, $Y_1$ and $Y_2$ the set $S$ is a regular tripod with edges
perpendicular to the sides of the triangle.
The sets $S_j$ are constructed so that $H^1(\tilde S_j) - H^1(S_j)$ 
is equal to the signed distance of $T$ from $Y_j$.
Notice also that the height of the triangle is equal to $1$.
So, by Lemma~\ref{lm:tripod} (with $Y$ in place of $Y_3$) 
we have
\[
\H^1(S) 
  =  \H^1([HT]) + \H^1(\tilde S_1) + \H^1(\tilde S_2) 
  = 1 + \H^1(S_1) + \H^1(S_2).
\]
Let us prove that $S_j\in \M(Y_j\cup A_j)$ for $j=1,2$.
Take any $\hat S_j\in \M(Y_j\cup A_j)$ 
for $j=1,2$ and consider $\hat S = \hat S_0 \cup \hat S_1 \cup \hat S_2$
where $\hat S_0$ is the tripod connecting $\hat S_1$ and $\hat S_2$ 
to $Y$ inside the triangle delimited by $Y$, $Y_1$ and $Y_2$.
We know that $\hat S_j$ touches $Y_j$ orthogonally hence,
again by Lemma~\ref{lm:tripod}, we have
$\H^1(\hat S_0) = 1$. Hence
\[
    \H^1(\hat S) 
    = 1 + \H^1(\hat S_1) + \H^1(\hat S_2)
    \le 1 + \H^1(S_1) + \H^1(S_2) = \H^1(S).
\]
Since $S\in \M(Y\cup A)$ while $\hat S \in \St(Y\cup A)$ we 
have $\H^1(S) \le \H^1(\hat S)$ hence the above inequalities 
are all equalities and $S_j\in M(Y_j\cup A_j)$ for $j=1,2$ as claimed.
\end{proof}














Let $Y_j:=f_j(Y)$ be the two lines with 60 degrees angles
with $Y$ and define $Y_i'$ as the line 
parallel to $Y_i$ and passing through $T$.

We are going to prove that $S_i\in \M(Y_i'\cup A_i)$ 
and then $S_i\in \M(Y_i,\cup A_i)$.

To prove that $S_j\in \M(Y_j'\cup A_j)$ we consider 
any minimizer $S_j' \in \M(Y_j'\cup A_j)$ 
and note that $S_j'$ has a terminal line segment
which touches $Y_j'$ orthogonally 
in a point $H_j'$. 
Let $T'$ be the intersection of the two lines 
containing the terminal line segments.
One can build a competitor $S'\in \St(Y\cup A)$
obtained by adding to $S_1'$ and $S_2'$ the triple 
junction composed by the three line segments joining 
$T'$ to the lines $Y$, $Y_1'$ and $Y_2'$.
By Lemma~\ref{lm:tripod} such a tripod 
has signed length equal to $\H^1(S_0)$
so that it turns out that 
$H^1(S') = \H^1(S_0) + \H^1(S_1') + \H^1(S_2')$.

By Lemma~\ref{lm:tripod} we know that 
$\H^1(S_0') = \H^1(S)$

An easy check shows that $\H^1(S_0') = \H^1(S_0)$.

Finally bla bla.....
\end{proof}

\begin{lemma}
  Let $S\in \M(Y\cup A)$ and $\lambda<\frac 1 {25}$.
  Let $S_0$, $S_1$ and $S_2$ be given as in the statement 
  of Lemma~\ref{lemma}
  
\end{lemma}

\begin{lemma}\label{lm:iteration}
  Let $S\in\M(Y\cup A)$ and suppose $S_0=[H,T]$, $S_1$ and $S_2$
  be as in Lemma~\ref{lm:branching}.
  Let $Y_j= f(Y)$ where $f(z) = 1+\lambda e^{(-1)^{j-1} i \frac \pi 3} z$ 
  for $j=1,2$.
  Let $T$ be the branching point of $S$ 
  and let $H_j$ be the projection of $T$ on $Y_j$, 
  for $j=1,2$. 
  Let $\tilde S_j = S_j\cup [H_j,T]$ 
  or $S_j \setminus [H_j,T]$
  depending on whether $T$ is on one side or 
  the other of $Y_j$.
  Then $\tilde S_j \in \M(Y_j\cup A_j)$ for $j=1,2$.
\end{lemma}
%
\begin{proof}
  Consider any $S_1'\in \M(Y_1\cup A_1)$ and $S_2'$ be the symmetric 
  of $S_1'$ with respect to symmetry axis of $A$.
  Let $[H_j',T_j']$ be the segment in $S_j'$ touching $Y_j$ in $H_j'$.
  Clearly $[H_j',T_j']$ is perpendicular to $Y_j$.
  Let $T'$ be the intersection of the line containing $[H_j',T_j']$
  with the line containing $Y$.
  If $T'$ is not a point of $S_1'$ then we consider the 
  triple junction $S_0'\in \M(\ENCLOSE{H_1',H_2',H})$ and
  define the competitor set 
  \[
    S' = S_0' \cup S_1' \cup S_2'.
  \]
  Otherwise we consider the segment 
  $S_0'=[T,T']$
\end{proof}

\begin{lemma}
  Let $S\in \M(Y\cup A)$ and let $H$ be the single point 
  in $H\cap $
\end{lemma}

\begin{lemma}
  If $S\in \M(Y\cup A)$ then $S\cap Y$ is composed by a single 
  point $H$ which lies on the symmetry axis of $A$.
\end{lemma}
\begin{proof}

\end{proof}

\end{document}